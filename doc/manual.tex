\documentclass[12pt]{amsart}
\usepackage{amssymb}
\usepackage{latexsym}
\usepackage{amscd}
\usepackage{epsf}
\usepackage{color}
\usepackage{amsopn,fixltx2e,microtype}
\usepackage{amsfonts}
\usepackage{euscript,mathrsfs} 
\usepackage{graphicx}    % standard LaTeX graphics tool
\usepackage{a4}
\usepackage{bbm}

\newcommand{\set}[1]{\left\lbrace #1 \right\rbrace} % standard set
\newcommand{\defas}{\mathrel{\mathop{:}}=}   % Definition
\renewcommand{\subset}{\subseteq}  % Subsetsymbols
\renewcommand{\supset}{\supseteq}
\newcommand{\todo}[1]{\{ \huge{Todo:}\normalsize #1 \}} % Todo
\DeclareMathOperator*{\bigtimes}{\textnormal{\Large $\times$}} % Cartesian Product
\DeclareMathOperator*{\supp}{supp}
\DeclareMathOperator*{\cl}{cl}
\DeclareMathOperator*{\spann}{span}
\newcommand{\comm}[1]{ }
\newcommand{\comment}[1]{ }
\providecommand{\abs}[1]{\left\lvert#1\right\rvert} 
\providecommand{\norm}[1]{\left\lVert#1\right\rVert}
\newcommand{\E}{\mathbb{E}} % expectation
\renewcommand{\P}{\mathbb{P}} % probability
\renewcommand{\d}{\mathrm{d}\:\!} %differential 
\newcommand{\e}{\mathrm{e}} % Eulers Number
\renewcommand{\i}{\mathrm{i}} % Imaginary Unit
\newcommand{\Ind}[1]{\mathbbm{1}_{\lbrace #1 \rbrace}} % indicator function
\newcommand{\Id}{\mathbbm{1}} % The identity map
\newcommand{\ie}{i.e.\;}  % make i.e. not the end of a sentence
\newcommand{\eg}{e.g.\;}  % make e.g. not the end of a sentence
\newcommand{\etc}{etc.\;}
\newcommand{\iid}{i.i.d.\;} % make i.i.d. not the end of sentence
\newcommand{\cipi}{\texttt{CIPI}\,}

\theoremstyle{plain}% default
\newtheorem{thm}{Theorem}
\newtheorem{lem}[thm]{Lemma}
\newtheorem{prop}[thm]{Proposition}
\newtheorem*{cor}{Corollary}

\theoremstyle{definition}
\newtheorem{defn}[thm]{Definition}
\newtheorem{conj}[thm]{Conjecture}
\newtheorem{exmp}[thm]{Example}

\theoremstyle{remark}
\newtheorem*{rem}{Remark}
\newtheorem*{note}{Note}
\newtheorem{case}{Case}
%%%%%%%%%%%%%%%%%%%%%%%%%%%%%%%%%%%%%%%%%%%%%%%%%%%%%%%%%%%%%%%%%%%%%

\begin{document}

\title[CIPI]{Computing Information Projections Iteratively with \cipi}

\author{L. Steiner and T. Kahle}
\address{Max Planck Institute for Mathematics in the Sciences, Inselstrasse 22, D-04103 Leipzig, Germany}
\email{\{kahle,steiner\}@mis.mpg.de}

\date{\today}

\comm{
\begin{abstract}
  some abstract\ldots
\end{abstract}
}

%% Maketitle here would insert a pagebreak
\maketitle

\section{Introduction}
In this document we describe the software tool \cipi, used to compute \emph{information
  projections}. background for what is done here can be found in~\cite{KahleDiplom06} while the
theory of information projections is described in~\cite{csiszarshields04}. The algorithm implemented
is \emph{iterative proportional fitting}.

Given an exponential family $\mathcal{E}$ of probability measures associated to a hypergraph on
$\set{1,\ldots,N}$ (a \emph{hierarchical model}) \cipi computes, for a given measure $\hat{P}$, the
minimizer of
\begin{equation*}
  D(\hat{P}\parallel Q) \qquad Q\in\overline{\mathcal{E}}, 
\end{equation*} where $D$ is the \emph{Kullback-Leibler Divergence}. It can be seen that this
minimizer equals the MLE of $\hat{P}$ in the closure of $\mathcal{E}$.

\begin{exmp}
  Consider the problem of contingency tables. Assume, in a survey 1000 people in
  Country A and Country B have been asked bout their smoking behaviour. This
  yielded the following results
  \begin{center}
  \begin{tabular}[]{|l|c|c|}
    \hline
    & male & female \\
    \hline
    smoking & 152 & 271 \\
    non-smoking & 319 & 258\\
    \hline
  \end{tabular}\hspace{1cm}
  \begin{tabular}[]{|l|c|c|}
    \hline
    & male & female \\
    \hline
    smoking & 219 & 331 \\
    non-smoking & 178 & 272\\
    \hline
  \end{tabular}
\end{center}

  A natural question to ask, is for independence of the factors by
  measuring the Kullback-Leibler Divergence of the empirical distribution to the
  ``closest'' approximation within the set of indipendent distributions, \ie the
  set 
  \begin{equation*}
    \mathcal{E}_1 \defas \set{P : P\left( x_1,x_2 \right)) = P_1(x_1)P_2(x_2)}
  \end{equation*}

  We now create two files called \texttt{A.dat}, \texttt{B.dat} 
  containing, one sample per line. So \texttt{A.dat} would contain 152 lines
  with the content ``00''. 271 lines with ``01'', 319 lines with ``10'' and 258
  lines containing ``11''. \cipi will build the emperical measure from the data.
  \texttt{B.dat} is filled correspondingly with the data of the second table. 

  Then we need to specify the parameters in a file called \texttt{PARAM}. We
  modify the example which is included, such that it has the follwing entries.
  \begin{verbatim}
  N=2
  Alphabet=01
  SetIterations=100
  InputType=Sample
  Hypergraph=
  OutputFilePrefix=A
  NumberOfProcesses=3
  \end{verbatim}
  Specifying an empty hypergraph will compute the projections to the exponential
  families of $k$-Interactions. \texttt{cipi} will output several files and the
  following messages.
  \begin{verbatim}
  writing p(0) to file A_p(0).txt
  writing p(1) to file A_p(1).txt
  writing p empirical to file A_p_empirical.txt
  2 : 0.0185144
  1 : 0.0135883

  run cipi in 0s
  read input in 0s
  calculate projection in 0s
\end{verbatim} We are
  primarily interested in the outputfile \texttt{A\_p(1).txt}. It contains the
  projection of the empirical measure (saved in \texttt{A\_p\_empirical.txt}) to
  the exponential family $\mathcal{E}_1$. The KL-distance of the empirical
  measure and this projection equals 0.0185144.
  
  When we run \cipi on \texttt{B.dat} we get the following output
  \begin{verbatim}
  writing p(0) to file A_p(0).txt
  writing p(1) to file A_p(1).txt
  writing p empirical to file A_p_empirical.txt
  2 : 3.56572e-06
  1 : 0.026379

  run cipi in 0s
  read input in 0s
  calculate projection in 0s
  \end{verbatim}

  In this case, the distance of the empirical measure to the familiy of
  independent distributions is very small. This indicates independence of the
  factors. 
\end{exmp}

\section{Usage}
\label{sec:usage}
\subsection{Parameters}
Every call of \cipi is must be followed by the name of a file containing parameters. This file
allows the following options
\begin{description}
  \item[N] The number of nodes.
  \item[SetIterations] The number of times every set of cardinality $k$ is visited in the projection
    loop.
  \item[InputType] Specifies how the data is read. Possible values are CharacterSequence(datapoint
    is a long sequence of symbols and the statistic is generated using a sliding window of lenght
    N), Sample(every line is a sample), Integer(every line is a sample and these samples are coded
    as integer) and Empirical(format of output files from \cipi).
  \item[Alphabet] specifies the alphabet which is used, \eg ``01'' or ``AGCT''.
  \item[Hpergraph] Describes the Hypergraph of the hierarchical model
  \item[OutputFilePrefix] Beginning of the file name used for the output files. 
  \item[NumberOfProcesses] Maximum number of processes running at the same time. 
\end{description}

\subsection{Output}
Currently \cipi prints to stdout and creates one file for each hypergraph that was considered. This
file contains the projected measure to that hypergraph.

\providecommand{\bysame}{\leavevmode\hbox to3em{\hrulefill}\thinspace}
\providecommand{\MR}{\relax\ifhmode\unskip\space\fi MR }
% \MRhref is called by the amsart/book/proc definition of \MR.
\providecommand{\MRhref}[2]{%
  \href{http://www.ams.org/mathscinet-getitem?mr=#1}{#2}
}
\providecommand{\href}[2]{#2}
\begin{thebibliography}{Kah06}

\bibitem[CS04]{csiszarshields04}
I.~Csisz{\'a}r and P.~C. Shields, \emph{Information theory and statistics: A
  tutorial}, Foundations and Trends in Communications and Information Theory,
  now Publishers, 2004.

\bibitem[Kah06]{KahleDiplom06}
Thomas Kahle, \emph{Information geometry and statistical learning theory},
  Master's thesis, University of Leipzig, 2006.

\end{thebibliography}

% \bibliographystyle{amsalpha}
% \bibliography{alotofstuff}

\end{document}





